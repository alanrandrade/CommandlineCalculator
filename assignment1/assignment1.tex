% Created 2016-09-26 Mon 12:12
\documentclass[11pt,a4paper]{article}
\usepackage[utf8]{inputenc}
\usepackage[T1]{fontenc}
\usepackage{fixltx2e}
\usepackage{graphicx}
\usepackage{longtable}
\usepackage{float}
\usepackage{wrapfig}
\usepackage{rotating}
\usepackage[normalem]{ulem}
\usepackage{amsmath}
\usepackage{textcomp}
\usepackage{marvosym}
\usepackage{wasysym}
\usepackage{amssymb}
\usepackage{hyperref}
\tolerance=1000
\bibliographystyle{plain}
\usepackage{microtype}
\usepackage{tikz}
\usepackage{circuitikz}
\usetikzlibrary{tikzmark,decorations.pathmorphing}
\author{Sasja Gillissen, Martin Huijben, Martijn Terpstra}
\date{\today}
\title{Title}
\hypersetup{
 pdfauthor={Sasja Gillissen, Martin Huijben, Martijn Terpstra},
 pdflang={English}}
\begin{document}

\maketitle
%\tableofcontents

\section{Introduction}
% Martin Hijben

%% State the objectives and overview of the document at a high-level.

\section{Test Goal}
% Sasja Gillissen

%% What is the overall goal of the testing effort, what are the final deliverables, who are the
%% stakeholders, i.e., for whom are you doing it, applicable laws and (international) standards.

\section{The Product}
% Sasja Gillissen

%% Identification of the SUT: What is the product (SUT – System Under Test) being tested, its
%% version, its operation context, required platform, its interfaces, and how is it executed.

\section{The Specification}
% Martijn Terpstra

%% What is the test basis, i.e., its specification, and all documentation describing what the SUT
%% shall do. (Do not include specification documents, but refer to them.)

the program will continuously prompt the user for input. It will
continue to prompt the user for input until it receives the exit
function as input and then terminates.

Depending on the input given the program will either

\begin{itemize}
\item Evaluate an expression
\item Assign a value to a variable and store it in memory
\item Assign a value to a function and store it in memory
\end{itemize}
The full specification is included in appendix\ref{app:specification}

%% What is the test basis, i.e., its specification, and all documentation describing what the SUT
%% shall do. (Do not include specification documents, but refer to them.)
\section{Risks}
% Martin Hijben

%% What are the risks of the product (at a high level), of the development process, and of the
%% test process. How are risks handled and mitigated.

\section{Test Environment}
% Sasja Gillissen

%% What is the (controlled) environment in which experiments are performed, what is the test
%% architecture, i.e., how are SUT and test system positioned and connected, which environment
%% and infrastructure (hardware, software, middleware, databases, libraries, . . .) are required for
%% testing, how to access the SUT and its interfaces, which stubs and drivers are needed, are
%% tests performed in a laboratory, production, or user environment.

\section{Quality Characteristics}
% Martin Hijben

%% Which quality characteristics are tested (IS 9126 or other quality model: functionality, reli-
%% ability, usability, . . .),

\section{Levels and Types of Testing}
% Martin Hijben

%% Which levels and types of tests are performed: (V-model: unit, integration, module, sys-
%% tem, acceptance, . . .), which units, components, subsystems, . . . are tested and for what,
%% accessibility (white/black box), verification vs. validation tests, . . ..

\section{Who will do the Testing}
% Martijn Terpstra

%% Who tests what, and what are the roles: developer, (independent) tester, user, alpha, certi-
%% fication, . . .),

\section{Test Generation Techniques}
% Sasja Gillissen

%% As far as already known or required, e.g., by applicable standards: black-box (equivalence
%% partitioning, boundary value analysis, error guessing, cause-effect graphing, decision tables,
%% state transitions, use case testing, exploratory testing, . . .), white-box (path, statement,
%% (multiple) condition, decision/branch, function, call, loop, MC/DC coverage, . . .), mutation
%% testing, combinatorial testing, . . ..

\section{Test Automation}
% Martijn Terpstra

%% As far as applicable, which parts of the testing will be automated, which test tools will be
%% used in the various phases of the testing process (planning, preparation, test generation, test
%% execution, completion), which tests are performed manually, what is automated, and which
%% tools have to be obtained or developed.

\section{Exit Criteria}
% Martin Hijben

%% What are the criteria for going from one test phase to the next, when is testing finished,
%% when is the product considered sufficiently tested, what are the (final) evaluation criteria.

\section{Testware}
% Martijn Terpstra

%% Which test products are recorded, consolidated, and kept for reuse.

\section{Issue Registration}
% Martijn Terpstra

%% How are issues (defects) registered, analysed, reported, and handled.

\appendix
\section{Specification} \label{app:specification}
\subsection{Overall use of the program}
the program will continuously prompt the user for input. It will
continue to prompt the user for input until it receives the
\texttt{exit()} function as input and then terminates.

On invalid input the program the program will show an error
indicating the nature of the invalid input, but will not terminate.

Valid input is on of the following
\begin{itemize}
\item A function assignment
\item A variables assignment
\item An expression
\item An equality test
\end{itemize}
\subsection{Memory}
The program will keep defined variables and function in memory
until the program terminates.


The program will be initialized with the following variables
\begin{itemize}
\item \texttt{pi}, whose value is \texttt{3.141592653589793}
\item \texttt{e}, whose value is \texttt{2.718281828459045}
\end{itemize}

The program will be initialized with the following functions
\begin{itemize}
\item \texttt{floor}, which rounds down
\item \texttt{ceil}, which rounds up
\item \texttt{round}, which rounds to the nearest  whole number.
\item \texttt{log}
\item \texttt{ln}, which
\item \texttt{sqrt}, which returns the square root of its
\item \texttt{root}, which returns the Nth root of it first argument.  (WARNING: does not work properly)
\item \texttt{exit} , which will terminate the program
\end{itemize}


\emph{BTW floor,ceil and round work incorrectly for NEGATIVE numbers}
\subsection{Expression}
A valid expression is defined using Context-free Grammar

Expression \(\rightarrow\) Seperator Expression Seperator

Expression \(\rightarrow\) (Expression)

Expression \(\rightarrow\) Number

Expression \(\rightarrow\) Variable

Expression \(\rightarrow\) PrefixFunction Seperator (Expression)

Expression \(\rightarrow\) Expression Seperator InfixFunction Seperator Expression

Seperator \(\rightarrow\) SPACE | \(\phi\) | Seperator Seperator

Number \(\rightarrow\) 0 | 1 | 2 | 3 | 4 | 5 | 6 | 7 | 8 | 9 | Number Number

Variable \(\rightarrow\) Name

PrefixFunction \(\rightarrow\) Name

Name \(\rightarrow\) SmallLetter | CapitalLetter | NameName

SmallLetter \(\rightarrow\) a|b|c|d|e|f|g|h|i|j|k|l|m|n|o|p|q|r|s|t|u|v|w|x|y|z

CapticalLetter \(\rightarrow\) A|B|C|D|E|F|G|H|I|J|K|L|M|N|O|P|Q|R|S|T|U|V|W|X|Y|Z

InfixFunction \(\rightarrow\) / | * | - | + | \^{}

\emph{variable names have a max length}


\subsection{Variable assignments}
A variable assignment is in the form

\texttt{VARIABLENAME = EXPRESSION}

Where a valid \texttt{VARIABLENAME} is a string of 1 or more letters and
is case-sensitive.
\subsection{Equality tests}
An equality test is in the form

\texttt{EXPRESSION1 = EXPRESSION2}

Where \texttt{EXPRESSION1} and \texttt{EXPRESSION2} are valid expressions and
\texttt{EXPRESSION1} is \textbf{not} the name of a variable

\subsection{Functions assignments}
A function assignment is in the form

\texttt{FUNCTIONNAME(ARGUMENT):=EXPRESSION}
\subsection{Expression parsing}
Intermediate steps and calculations are show
The result of the expression is shown

For instance an expression in the form \texttt{(1 + 2) * 4} will require the following actions.

\begin{itemize}
\item Evaluate \texttt{1 + 2} and print its result, (\texttt{3} in this case).
\item Evaluate \texttt{3 * 4} and print its result, the \texttt{3} being the result of the previous evaluation.
\item Print the results of the entire expression, \texttt{12 in this case}, after all intermediate calculations have been done.
\end{itemize}

\end{document}
