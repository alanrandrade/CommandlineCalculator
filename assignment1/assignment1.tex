\documentclass[11pt,a4paper]{article}
\usepackage[utf8]{inputenc}
%\usepackage[T1]{fontenc}
\usepackage{graphicx}
\usepackage{longtable}
\usepackage{float}
\usepackage{wrapfig}
\usepackage{rotating}
\usepackage[normalem]{ulem}
\usepackage{amsmath}
\usepackage{textcomp}
\usepackage{marvosym}
\usepackage{wasysym}
\usepackage{amssymb}
\usepackage{hyperref}
\tolerance=1000
\bibliographystyle{plain}
\usepackage{microtype}
\usepackage{tikz}
\usepackage{circuitikz}
\usetikzlibrary{tikzmark,decorations.pathmorphing}
\author{Sasja Gillissen, Martin Huijben, Martijn Terpstra}
\date{\today}
\title{Testing Techniques\\
  \textbf{Assignment 1}}
\hypersetup{
 pdfauthor={Sasja Gillissen, Martin Huijben, Martijn Terpstra},
 pdflang={English}}


\begin{document}

\maketitle
%\tableofcontents

\section{Introduction}
% Martin Huijben
In this document we will describe
bla bla intro
%% State the objectives and overview of the document at a high-level.

\section{Test Goal}
% Sasja Gillissen

%% What is the overall goal of the testing effort, what are the final deliverables, who are the
%% stakeholders, i.e., for whom are you doing it, applicable laws and (international) standards.

\section{The Product}
% Sasja Gillissen
The system under test is a calculator with a command line interface (cli). This system is developed in Java and has no external dependencies. This product can be executed on all platforms supported by the Java Virtual Machine (JVM). The user can interface with the program using the command line, upon executing the program, it will show a prompt in which the user can type expressions or define variables and functions. The input will be parsed, evaluated and the result will be printed. This process repeats itself until te user quits the program.

%% Identification of the SUT: What is the product (SUT - System Under Test) being tested, its
%% version, its operation context, required platform, its interfaces, and how is it executed.

\section{The Specification}
% Martijn Terpstra

%% What is the test basis, i.e., its specification, and all documentation describing what the SUT
%% shall do. (Do not include specification documents, but refer to them.)

the program will continuously prompt the user for input. It will
continue to prompt the user for input until it receives the exit
function as input and then terminates.

Depending on the input given the program will either

\begin{itemize}
\item Evaluate an expression
\item Assign a value to a variable and store it in memory
\item Assign a value to a function and store it in memory
\end{itemize}
The full specification is included in appendix\ref{app:specification}

%% What is the test basis, i.e., its specification, and all documentation describing what the SUT
%% shall do. (Do not include specification documents, but refer to them.)
\section{Risks}
% Martin Huijben
This product is advertised as a working calculator with function and variable memory. Every time the product does not comply with these expectations, this causes a risk, since people might not double check the answers, have another calculator or have a backup of the inserted data. This makes the following scenarios risks:
\begin{itemize}
	\item The calculator gives a wrong answer. A risk since the user might not notice this.
	\item The calculator crashes. A risk since valuable function and variable memory can be deleted.
	\item The calculator gives an error. A minor risk since the user knows the product can not comply, no data is lost and often there is another calculator.
\end{itemize}
Finding those scenarios will be a main focus in our testing.

There are no risks in the development process. The development process is already finished.

Risks in the test process are scenarios which will make the test process to lengthy. Since we know an average test case will not take long (typing a few lines), the only risky scenario is when we have too many test cases. If this happens we will either start automating or lower our goal of testing every part. That being said, we do not expect this to happen.
%% What are the risks of the product (at a high level), of the development process, and of the
%% test process. How are risks handled and mitigated.

\section{Test Environment}
% Sasja Gillissen
The tests will be performed on a single PC with JRE 1.8. The SUT will be tested using manual tests. No other software is required.
%% What is the (controlled) environment in which experiments are performed, what is the test
%% architecture, i.e., how are SUT and test system positioned and connected, which environment
%% and infrastructure (hardware, software, middleware, databases, libraries, . . .) are required for
%% testing, how to access the SUT and its interfaces, which stubs and drivers are needed, are
%% tests performed in a laboratory, production, or user environment.

\section{Quality Characteristics}
% Martin Huijben

%% Which quality characteristics are tested (IS 9126 or other quality model: functionality, reli-
%% ability, usability, . . .),

If we follow the quality characteristics of ISO 9126, then functionality and reliability would be our main focus. Here follows a per characteristic description of if we need testing for it:
\begin{description}
	\item[Functionality] will be tested thoroughly. We have a complete and detailed specification of the SUT, to which the SUT should comply.
	\item[Reliability] will be a key point. Fault tolerance and recoverability are important to us, for the reason that on this area the SUT does not comply with our initial expectations. Some simple experimenting showed that our SUT could crash at some input, instead of just giving an error. We want to test in which cases this happens.
	\item[Usability] is of little concern to us. Since the SUT is meant as a commandline calculator and is not meant for the larger public, the usability expectations are low. The SUT already surpasses our expectations.
	\item[Efficiency] will not be tested. The SUT is so small that to the user every calculation will seem as instantaneous, regardless of the actual efficiency.
	\item[Maintainability] will not be tested. The difficulty to adapt the SUT does not interest us.
	\item[Portability], the last characteristic. This will also not be tested. The reason for this is because we already know the portability. Every computer with command line can use the SUT.
\end{description}

\section{Levels and Types of Testing} \label{levels}
% Martin Huijben

The SUT can be split into clear units. There is one unit for each predefined variable and function. Furthermore we have units for new variables, equality tests, new functions and expression parsing. These units can be tested separately and in combinations. Together they form the system, with its specification as counterpart.

We are only interested in the capabilities of our SUT, not in how one would fix found shortcomings. Therefore we do not test white box. We will be doing validation testing, using the specification.

%% Which levels and types of tests are performed: (V-model: unit, integration, module, sys-
%% tem, acceptance, . . .), which units, components, subsystems, . . . are tested and for what,
%% accessibility (white/black box), verification vs. validation tests, . . ..

\section{Who will do the Testing}
% Martijn Terpstra

Test will be performed by independent testers. This is because little
knowledge is needed to preform the test and those already familiar
with the program may be biased.

%% Who tests what, and what are the roles: developer, (independent)
%% tester, user, alpha, certification, . . .),

\section{Test Generation Techniques}
% Sasja Gillissen
The Black-Box tests will first be generated by error guessing based on the specification. The test suite will be expanded using partitioning of the input values, and applying boundary value analysis on these partitions.

%% As far as already known or required, e.g., by applicable standards: black-box (equivalence
%% partitioning, boundary value analysis, error guessing, cause-effect graphing, decision tables,
%% state transitions, use case testing, exploratory testing, . . .), white-box (path, statement,
%% (multiple) condition, decision/branch, function, call, loop, MC/DC coverage, . . .), mutation
%% testing, combinatorial testing, . . ..

\section{Test Automation}\label{sec:test-automation}
% Martijn Terpstra

Tests will be done manually. Each test is described in a spreadsheet.

To start a test, a new spreadsheet is copied from a template.

This template contains:

\begin{itemize}
\item One or more lines of input to be entered in the program
\item A description of the expected output
\item A field to insert the resulting output after entering the input.
\end{itemize}

The template is updated if and only if the tests  changed.

If the resulting output is equal to the expected output, the test is
successful. If not the test is unsuccessful.

The results of the test are recorded in the the spreadsheet. A new
spreadsheet is created for each testing session.

After the testing session is over, the resulting spreadsheet shall be
send via email to the developer team for review.

%% As far as applicable, which parts of the testing will be automated, which test tools will be
%% used in the various phases of the testing process (planning, preparation, test generation, test
%% execution, completion), which tests are performed manually, what is automated, and which
%% tools have to be obtained or developed.

\section{Exit Criteria}
% Martin Huijben

In the section \ref{levels} we told about the units our SUT could be split in. We consider every combination of units as a part that should be tested. Per such a part we will consider ourselves finished if we feel certain that we can predict every outcome for every input. Since the SUT is defined in a context-free grammar, this certainty can be achieved. When all parts have been tested in such a way, we consider ourselves finished.

%% What are the criteria for going from one test phase to the next, when is testing finished,
%% when is the product considered sufficiently tested, what are the (final) evaluation criteria.

\section{Testware}
% Martijn Terpstra

As described in section \ref{sec:test-automation}, the results, with
the description of the test are written to a spreadsheet. The
resulting spreadsheet is the test product and shall be stored in a
digital format.

%% Which test products are recorded, consolidated, and kept for reuse.

\section{Issue Registration}
% Martijn Terpstra

After a testing as described in section \ref{sec:test-automation} the
developer team receives an email with a spreadsheet attached.

This spreadsheet will inform the developer if any tests have failed
and if so what tests have failed and what happened when they failed.

The developer then can make bug reports based on unsuccessful test and
fix them in the future.

%% How are issues (defects) registered, analysed, reported, and handled.

\appendix
\section{Specification} \label{app:specification}
\subsection{Overall use of the program}
the program will continuously prompt the user for input. It will
continue to prompt the user for input until it receives the
\texttt{exit()} function as input and then terminates.

On invalid input the program the program will show an error
indicating the nature of the invalid input, but will not terminate.

Valid input is on of the following
\begin{itemize}
\item A function assignment
\item A variables assignment
\item An expression
\item An equality test
\end{itemize}
\subsection{Memory}
The program will keep defined variables and function in memory
until the program terminates.


The program will be initialized with the following variables
\begin{itemize}
\item \texttt{pi}, whose value is \texttt{3.141592653589793}
\item \texttt{e}, whose value is \texttt{2.718281828459045}
\end{itemize}


The program will be initialized with the following functions
\begin{itemize}
\item \texttt{floor}, which rounds down
\item \texttt{ceil}, which rounds up
\item \texttt{round}, which rounds to the nearest  whole number.
\item \texttt{log}
\item \texttt{ln}, which
\item \texttt{sqrt}, which returns the square root of its
\item \texttt{root}, which returns the Nth root of it first argument.  (WARNING: does not work properly)
\item \texttt{exit} , which will terminate the program
\end{itemize}



\emph{BTW floor,ceil and round work incorrectly for NEGATIVE numbers}
\subsection{Expression}
A valid expression is defined using Context-free Grammar

Expression \(\rightarrow\) Separator Expression Separator

Expression \(\rightarrow\) (Expression)

Expression \(\rightarrow\) Number

Expression \(\rightarrow\) Variable

Expression \(\rightarrow\) PrefixFunction Separator (ArgumentList)

Expression \(\rightarrow\) Expression Separator InfixFunction Separator Expression

ArgumentList \(\rightarrow\) \(\phi\) $\mid$  NonEmptyArgumentList

NonEmptyArgumentList  \(\rightarrow\)  Expression $\mid$ Expression , NonEmptyArgumentList

Separator \(\rightarrow\) SPACE $\mid$ \(\phi\) $\mid$ Separator Separator

Number \(\rightarrow\) 0 $\mid$ 1 $\mid$ 2 $\mid$ 3 $\mid$ 4 $\mid$ 5 $\mid$ 6 $\mid$ 7 $\mid$ 8 $\mid$ 9 $\mid$ Number Number

Variable \(\rightarrow\) Name

PrefixFunction \(\rightarrow\) Name

Name \(\rightarrow\) SmallLetter $\mid$ CapitalLetter $\mid$ NameName

SmallLetter \(\rightarrow\) a$\mid$b$\mid$c$\mid$d$\mid$e$\mid$f$\mid$g$\mid$h$\mid$i$\mid$j$\mid$k$\mid$l$\mid$m$\mid$n$\mid$o$\mid$p$\mid$q$\mid$r$\mid$s$\mid$t$\mid$u$\mid$v$\mid$w$\mid$x$\mid$y$\mid$z

CapticalLetter \(\rightarrow\) A$\mid$B$\mid$C$\mid$D$\mid$E$\mid$F$\mid$G$\mid$H$\mid$I$\mid$J$\mid$K$\mid$L$\mid$M$\mid$N$\mid$O$\mid$P$\mid$Q$\mid$R$\mid$S$\mid$T$\mid$U$\mid$V$\mid$W$\mid$X$\mid$Y$\mid$Z

InfixFunction \(\rightarrow\) / $\mid$ * $\mid$ - $\mid$ + $\mid$ \^{}

\emph{variable names have a max length}


\subsection{Variable assignments}
A variable assignment is in the form

\texttt{VARIABLENAME = EXPRESSION}

Where a valid \texttt{VARIABLENAME} is a string of 1 or more letters and
is case-sensitive.
\subsection{Equality tests}
An equality test is in the form

\texttt{EXPRESSION1 = EXPRESSION2}

Where \texttt{EXPRESSION1} and \texttt{EXPRESSION2} are valid expressions and
\texttt{EXPRESSION1} is \textbf{not} the name of a variable

\subsection{Functions assignments}
A function assignment is in the form

\texttt{FUNCTIONNAME(ARGUMENTS):=EXPRESSION}

where ARGUMENTS is an ArgumentList.
\subsection{Expression parsing}
Intermediate steps and calculations are show
The result of the expression is shown

For instance an expression in the form \texttt{(1 + 2) * 4} will require the following actions.

\begin{itemize}
\item Evaluate \texttt{1 + 2} and print its result, (\texttt{3} in this case).
\item Evaluate \texttt{3 * 4} and print its result, the \texttt{3} being the result of the previous evaluation.
\item Print the results of the entire expression, \texttt{12 in this case}, after all intermediate calculations have been done.
\end{itemize}

\end{document}

%%  LocalWords:  Terpstra Sasja Gillissen Huijben tikzmark pdfauthor
%%  LocalWords:  pdflang PrefixFunction SmallLetter CapitalLetter
%%  LocalWords:  NameName CapticalLetter VARIABLENAME FUNCTIONNAME
%%  LocalWords:  ArgumentList NonEmptyArgumentList
